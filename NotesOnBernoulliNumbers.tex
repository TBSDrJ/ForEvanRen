\documentclass[12pt]{article}

\usepackage{amsmath, hyperref}

\topmargin=0in
\headheight=0in
\headsep=0in
\oddsidemargin=0in
\textwidth=6.5in
\textheight=8.5in

\title{Notes on Bernoulli Numbers for Evan}
\author{Dr. J}

\begin{document}

\maketitle

\section{Quick Note on subscripts}

Bernoulli numbers have been subscripted in three different ways:
\begin{enumerate}
	\item
Where $B_1 = -\frac{1}{2}$.  This is now standard (?), and is what I use in my dissertation.  You probably learned it this way.  If not, let me know.  That's what I'll use unless I hear otherwise from you.
	\item
Where $B_1 = \frac{1}{2}$.  This was common in algebraic number theory pre-1980, e.g. Iwasawa or Washington.  $B_n$ for $n \neq 1$ agrees entirely with 1., above, so it only differs in this one location.
	\item
Where $B_1 = \frac{1}{6}$.  In this case, $B_n$ of this kind = $B_{2n}$ of either 1. or 2., above.  This was common in Algebraic Topology, e.g. Milnor (one of the true giants of $20^{\text{th}}$-century mathematics, still alive and well in NJ as far as I know) and Lance (my advisor, who you see mentioned several times in my dissertation).
\end{enumerate}

In a line of the Preface that made me laugh out loud when I read it (though I doubt it will for you), \underline{Introduction to Cyclotomic Fields}, by Larry Washington of University of Maryland said:``At Serge Lang's urging I have let the first Bernoulli number be $B_1 = -\frac{1}{2}$ rather than $+\frac{1}{2}$.  This disagrees with Iwasawa [Washington's advisor at Princeton] and several of my papers, but conforms to what is becoming standard usage."  Serge Lang was well-known for churning out huge textbooks in almost any field of graduate-level mathematics, whether he was an expert in that field or not.  (This is likely connected to Lang's membership in the Bourbaki, which, if you don't know the story of Nicholas Bourbaki, you should look it up or ask me).  So of course Lang would have done this.  Lang also famously traveled with a delegation to the Republic of South Africa where many thousands of people were dying in an AIDS epidemic; they successfully convinced the government there that the HIV virus did not cause AIDS, and that preventing transmission of HIV would not slow the epidemic.  This was disastrous, and the policies of the RSA government following this resulted in much loss of life there.

Just banging around doing a little research, it seems that this is not so fully resolved/ standard as I had thought, and is still in debate: \href{http://luschny.de/math/zeta/The-Bernoulli-Manifesto.html}{link to Conversation between Peter Luschny and Donald Knuth on this topic}.  Note that Donald Knuth is the same guy who, back in the 1970's, got fed up with trying to type a mathematical paper for submission to a journal that he decided to create the first version of \TeX\ so that he could properly typeset it.

\end{document}